\documentclass{article}
\usepackage{xcolor}
\usepackage{listings}
\usepackage{courier}  % Use proper monospace fonts

% Define colors
\definecolor{backcolour}{HTML}{272823}
\definecolor{comment}{HTML}{878471}
\definecolor{string}{HTML}{DFD780}
\definecolor{keyword}{HTML}{B3E053}
\definecolor{number}{HTML}{A783F7}
\definecolor{white}{RGB}{255,255,255}

% Listings style
\lstdefinestyle{python}{
    basicstyle=\ttfamily\color{white},  % Monospace font
    backgroundcolor=\color{backcolour}, % Smooth background
    commentstyle=\color{comment},
    keywordstyle=\color{keyword},
    identifierstyle=\color{white},
    stringstyle=\color{string},
    numbers=none,  % Disable line numbers
    frame=single,
    framerule=0pt,
    showspaces=false,
    showstringspaces=false,
    literate=
      *{0}{{{\color{number}0}}}{1}
       {1}{{{\color{number}1}}}{1}
       {2}{{{\color{number}2}}}{1}
       {3}{{{\color{number}3}}}{1}
       {4}{{{\color{number}4}}}{1}
       {5}{{{\color{number}5}}}{1}
       {6}{{{\color{number}6}}}{1}
       {7}{{{\color{number}7}}}{1}
       {8}{{{\color{number}8}}}{1}
       {9}{{{\color{number}9}}}{1}
}

% Apply style globally
\lstset{style=python}

\begin{document}

\begin{lstlisting}
# Example Python code
def add(a, b):
    return a + b

print(add(3, 5))  # Should print 8

for i in range(10):
    print(i)
\end{lstlisting}

\end{document}

\definecolor{blurple}{HTML}{5865F2}
\definecolor{backcolour}{HTML}{272823}
\definecolor{comment}{HTML}{878471}
\definecolor{string}{HTML}{DFD780}
\definecolor{keyword}{HTML}{B3E053}
\definecolor{keyword2}{HTML}{E53F73}
\definecolor{number}{HTML}{A783F7}

\lstdefinestyle{python}{
    basicstyle=\color{white}\ttfamily,
    backgroundcolor=\color{backcolour},
    commentstyle=\color{comment},
    identifierstyle=\color{white},
    stringstyle=\color{string},
    keywordstyle=\color{keyword2}, 
    keywordstyle=[2]\color{keyword},
    moredelim=[s][\color{keyword}]{np}{.}
}


\lstset{
    style=python,
    literate=
        {0}{{{\color{number}0}}}{1}
        {1}{{{\color{number}1}}}{1}
        {2}{{{\color{number}2}}}{1}
        {3}{{{\color{number}3}}}{1}
        {4}{{{\color{number}4}}}{1}
        {5}{{{\color{number}5}}}{1}
        {6}{{{\color{number}6}}}{1}
        {7}{{{\color{number}7}}}{1}
        {8}{{{\color{number}8}}}{1}
        {9}{{{\color{number}9}}}{1}
        {=}{{{\color{keyword2}=}}}{1}
        {+}{{{\color{keyword2}+}}}{1}
        {-}{{{\color{keyword2}-}}}{1}
        {*}{{{\color{keyword2}*}}}{1}
        {/}{{{\color{keyword2}/}}}{1}
        {<}{{{\color{keyword2}<}}}{1}
        {>}{{{\color{keyword2}>}}}{1}
}

\begin{minted}{python}
    import numpy as np

    def wow(a):
        return a

    for i in range(0,10):
        x = 'i'
        y = 1+i
        z = np.pi
\end{minted}