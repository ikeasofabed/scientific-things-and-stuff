\documentclass[12pt]{article}
\usepackage[a4paper, margin=2cm]{geometry}
\usepackage[english]{babel} % To obtain English text with the blindtext package
\usepackage{blindtext}
\usepackage{graphicx} % Required for inserting images
\usepackage{array, multirow} % For extra column formatting
\usepackage{amsmath, amssymb, cancel} %for equation environment
\usepackage{float}
\usepackage{parskip} % For gaps between para
\usepackage{setspace}
\usepackage{pdfpages}
\usepackage[export]{adjustbox}
\usepackage{emptypage}
\usepackage{tocloft}
\usepackage[nottoc]{tocbibind}
\usepackage{hyperref, url}
\usepackage[table]{xcolor}

\cftsetindents{section}{0em}{2em}
\cftsetindents{subsection}{0em}{2em}

\renewcommand\cfttoctitlefont{\hfill\Large\bfseries}
\renewcommand\cftaftertoctitle{\hfill\mbox{}}

\graphicspath{ {./images/} }

\definecolor{blurple}{HTML}{5865F2}
\definecolor{backcolour}{HTML}{272823}

\hypersetup{
    colorlinks=true,
    linkcolor=black,
    urlcolor=black,
    citecolor=blurple,
}

\urlstyle{same}

\renewcommand{\arraystretch}{1.3}

\setcounter{secnumdepth}{5}
\setcounter{tocdepth}{5}
\newcommand\simpleparagraph[1]{%
  \stepcounter{paragraph}\paragraph*{\theparagraph\quad{}#1}}

%%%%%%%%%%%%%%%%%%%%%%%%%%%%%%%%%%%


\title{PHYC20090 Reflection}
\author{Joana Adao}
\date{\today}


%%%%%%%%%%%%%%%%%%%%%%%%%%%%%%%%%%%

\begin{document}

\thispagestyle{empty}

I learned that the interference pattern formed with the \textbf{Michelson Interferometer} is highly dependent on the mirror position and tilt as any minor change could result in path length differences, altering the phase
shift of the light and therefore the interference pattern shape and curve. I was able to see this effect in action during my experiment through slowly tilting the mirror in a different direction and seeing how the interference pattern
moved with this. I also found that the \textbf{Hall Effect} functions  similarly; tiny changes in the Hall Voltage would alter the magnetic field strength and served to tell me methe charge carriers and their density.

These tiny shifts found in both experiments reminded me of Maxwell's equations, and how they too are varied on incremental change, typically of time. While the experiments aren't tied to a single equation,
I found they helped me understand them physically. I was able to connect the Hall Effect to Faraday's Law, where changes in the magnetic field over time induced electric fields. 
And even with the Interferometer,I could still see how the small changes in mirror position led to the big observable changes. When you start looking at the bigger picture described, the sensitive systems start to make more sense.

These experiments got me thinking on what else might be able to be detected through small changes. 
Could there be a way to detect particles, like the axions or Majorana fermions, that have been misunderstood because we haven't been studying them with the correct methods?
Is there a chance they might cause a slight change in the interference pattern of light through a subtle phase shift, or slightly affect a magnetic field due to a small change in the Hall voltage?
I don't know if it would be possible using these exact approaches, but perhaps considering exploring other miniscule changes to otherwise stable surroundings could uncover new information.

\end{document}