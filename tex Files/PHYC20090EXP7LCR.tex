\documentclass[12pt]{article}
\usepackage[a4paper, margin=2cm]{geometry}
\usepackage[english]{babel} % To obtain English text with the blindtext package
\usepackage{blindtext}
\usepackage{graphicx} % Required for inserting images
\usepackage{array} % For extra column formatting
\usepackage{amsmath} %for equation environment
\usepackage{float}
\usepackage{parskip} % For gaps between para
\usepackage{setspace}
\usepackage{pdfpages}
\usepackage{abstract}
\usepackage[export]{adjustbox}
\usepackage{emptypage}
\usepackage{tocloft}
\usepackage[nottoc]{tocbibind}
\usepackage{url}


\cftsetindents{section}{0em}{2em}
\cftsetindents{subsection}{0em}{2em}

\renewcommand\cfttoctitlefont{\hfill\Large\bfseries}
\renewcommand\cftaftertoctitle{\hfill\mbox{}}

\graphicspath{ {./images/} }

\pagenumbering{arabic}


%%%%%%%%%%%%%%%%%%%%%%%%%%%%%%%%%%%


\title{PHYC20090 Exp.7 LCR Circuits}
\author{Joana Adao}
\date{\today}

\begin{document}

\begin{titlepage}
    \begin{center}

        \begin{figure}[ht]
            \includegraphics[width=\textwidth]{UCDLogo.png}
        \end{figure}
        
        \begin{figure}
            \centerline{\includegraphics[width=\paperwidth]{UCDBanner.png}}
        \end{figure}

        \vspace{4cm}

        {\Huge \bfseries PHYC20090 Electronics and Devices}\\
        \vspace{0.75cm}
        {\LARGE Experiment No.7 Sinusodial Response of the LCR Resonant Circuit }
        
        \vspace{1cm}
    
    {\Large \textbf{27 January 2025 }}

    \vspace{2cm}
    
    {\large \textbf{by Joana C.C. Adao (Student No. 23311051)}}\\
    \medskip
    {\large With Arminas A., Ananya L., Samuel S.}

    \end{center}
    
   \clearpage

\end{titlepage}

\setcounter{page}{1}
\tableofcontents

\newpage

\begin{abstract}
\addcontentsline{toc}{section}{Abstract}

The aim of this experiment was to



\end{abstract}


%%%%%%%%%%%%%%%%%%%%%%%%%%%%%%%%%%%


\section{Theory}

\subsection{LCR Circuits}
An LCR circuit is made up of inductors (L), capacitors (C), and resistors (R), usually connected in series.
Since all the components of the circuit are connected in series, equal amount of the current will flow through each element.
\cite{unacademy}

A circuit containing these components, L, C, and R, can act as themselves individually at certain frequencies
\cite{learnabout}, §1.2.1.
The LCR circuit can also magnify the voltages across the L, C, and R such that it is larger than the  circuit's input voltage (ie. AC)
\cite{learnabout}.

\subsubsection{Inductance, Capacitance, Resistance}

Inductance, capacitance, and resistance make up the basic parameters that can affect circuits up to some degree
\cite{elecnotes}.

\textbf{Inductance} is a property of a conductor and its 
\cite{britinductance}


\subsection{Wave Properties}





\subsubsection{Resonance}






\section{The Procedure}



\section{Results and Calculations}



\section{Conclusion}



\section{Expansion on the Experiment}




\newpage


%%%%%%%%%%%%%%%%%%%%%%%%%%%%%%%%%%%

\bibliographystyle{IEEEtran}
\bibliography{References}


\newpage

\section*{Appendix}
\addcontentsline{toc}{section}{Appendix}

\subsection*{Raw Data}
\addcontentsline{toc}{subsection}{Raw Data}




\end{document}