\documentclass[11pt, a4paper, twocolumn]{article}

% Fonts and math (Elsevier-like Times style)
\usepackage{graphicx} % Required for inserting images
\usepackage{array} % For extra column formatting
\usepackage{amsmath, amssymb, cancel} % for equation environment
\usepackage{float,geometry}
\geometry{margin=2cm}
\usepackage[skip=10pt]{parskip}
\usepackage{setspace}
\usepackage{hyperref}
\usepackage{cite, bookmark}
\usepackage{url}
\usepackage[table,xcdraw]{xcolor}
\usepackage[export]{adjustbox}
\usepackage{listings}

% --- Abstract styling (elsarticle look)
\usepackage{abstract}
\renewcommand{\abstractnamefont}{\normalfont\bfseries}
\renewcommand{\abstracttextfont}{\normalfont}
\renewenvironment{abstract}{%
  \par\noindent\rule{\linewidth}{0.4pt}\par
  \begin{center}\bfseries Abstract\end{center}%
}{%
  \par\rule{\linewidth}{0.4pt}\par
}

% --- Section styling (similar to elsarticle
\usepackage{titlesec}
\titleformat{\section}{\normalfont\large\bfseries}{\thesection.}{0.5em}{}
\titleformat{\subsection}{\normalfont\normalsize\bfseries}{\thesubsection.}{0.5em}{}

% Table of contents styling
\usepackage{tocloft}
\cftsetindents{section}{0em}{2em}
\cftsetindents{subsection}{1em}{2em}
\renewcommand\cfttoctitlefont{\hfill\Large\bfseries}
\renewcommand\cftaftertoctitle{\hfill\mbox{}}

\graphicspath{ {./images/} }

\definecolor{blurple}{HTML}{5865F2}
\definecolor{backcolour}{HTML}{272823}

\hypersetup{
    colorlinks=true,
    linkcolor=black,
    urlcolor=black,
    citecolor=blurple,
}

\urlstyle{same}

\DeclareMathOperator{\sinc}{sinc}

\pagenumbering{arabic}
\renewcommand{\arraystretch}{1.4}

% Python code export
\lstset{
    language=Python,
    basicstyle=\ttfamily\scriptsize,
    keywordstyle=\color{teal},       
    commentstyle=\color{gray},        
    stringstyle=\color{magenta},  
    showstringspaces=false,
    frame=single,
    breaklines=true,
    numbers=left,
    numberstyle=\tiny\color{gray}
}

\setcounter{secnumdepth}{5}
\setcounter{tocdepth}{5}

%%%%%%%%%%%%%%%%%%%%%%%%%%%%%%%%%%%

\title{%

\includegraphics[width=0.2\linewidth]{ucd_brandmark_black.jpg} \vspace{0.5cm}\\

{\Large\bfseries UCD School of Physics}\vspace{1.5cm}\\

{\large PHYC30170 Physics Astronomy and Space Lab I}\\
{\bfseries Electronics Laboratory}

}
\author{Joana C.C. Adao \\ \small Student No.: 23311051}
\date{18 October 2025}

\begin{document}

% Title page - single column
\onecolumn
\maketitle

% Abstract - single column
\thispagestyle{empty}
\begin{abstract}

This is the abstract.

\end{abstract}

\newpage

% Table of Contents - single column
\setcounter{page}{1}
\tableofcontents

\newpage

%%%%%%%%%%%%%%%%%%%%%%%%%%%%%%%%%%%

% Start main content in two columns
\twocolumn

\section{Introduction} \label{sec:1}

\section{Experiment 5} \label{sec:2}

The objective of this experiment is to simulate and build a simple RC circuit driven by a square wave voltage source.
The voltage across the resistor and capacitor will be observed using an oscilloscope, and the time constant of the circuit will be determined.

\subsection{Theory} \label{sec:2.1}

\subsubsection{Kirchhoff's Laws} \label{sec:2.1.1}

Kirchhoff has two primary and distinct laws that deal with the conservation of charge and energy within electrical circuits. These are:

\textbf{Kirchhoff's Current Law (KCL):} This law states that the sum of currents entering a junction (or node) in an electrical circuit 
must equal the sum of currents leaving that junction. Essentially, it is a statement of the conservation of electric charge.

\textbf{Kirchhoff's Voltage Law (KVL):} This law states that the sum of electrical charge, or the voltage drop,
around any closed loop is zero. Essentially, it is a statement of the conservation of energy within an electrical circuit. \cite{horowitz1989art}

\subsubsection{Time Constant} \label{sec:2.1.2}

The time constant, denoted by the Greek letter tau (\( \tau \)), is a measure of the time it takes for a system to respond to a change in input. 
For electrical circuits, the time constant is often used to describe the behavior of an RC (resistor-capacitor) circuit.

The time constant for an RC circuit is given by the formula:

\begin{equation} \label{eq:1}
    \tau = R C
\end{equation}

And such, the solution to the equation for a circuit's voltage response over time is then given by:

\begin{equation} \label{eq:2}
    V(t) = V_f + Ae^{-\frac{t}{\tau}}
\end{equation}

where R is the resistance in ohms (\( \Omega \)), C is the capacitance in farads (F), \( V(t) \) is the voltage at time t,
\( V_f \) is the final voltage, A is a constant determined by initial conditions. \cite{horowitz1989art}

As the capacitor charges, the slope (which is proportional to the current) is proportional to the remaining voltage. Therefore, the
waveform produced is exponential in nature. For \( t \gg \tau \), the voltage across the capacitor approaches its final value \( V_f \) asymptotically. \cite{horowitz1989art}

\subsection{Methodology} \label{sec:2.2}

The circuit was constructed on the TINA software as shown in Figure \ref{fig:5}. 
A resistor of 1 k\( \Omega \) and a capacitor of 1 \( \mu \)F were connected in series in the circuit.
A square wave voltage source was used to drive the circuit, with a frequency of 1 kHz and an amplitude of 5 V.
The voltage across the capacitor and resistor was observed using an oscilloscope.

\begin{figure}[H]
    \centering
    \includegraphics[width=.9\linewidth]{ex5_setup.JPG}
    \caption{Experimental setup constructed on the TINA software.}
    \label{fig:5}
\end{figure}

A similar physical circuit was then constructed on a veroboard using the same components as in the simulation.
The oscilloscope was used to supply the square wave voltage source and to observe the voltage across the capacitor and resistor.

\subsection{Results} \label{sec:2.3}

Results were gathered from both the TINA simulation and the physical circuit using the oscilloscope. The graphs produced are shown in
Figures \ref{fig:5.1} and \ref{fig:5.2}.

\begin{figure}[ht]
    \centering
    \includegraphics[width=\linewidth]{ex5_TINA.png}
    \caption{Oscilloscope output from the TINA simulation driven by a square wave voltage source.}
    \label{fig:5.1}
\end{figure}

The oscilloscope shows the two channels: Channel 1 (blue) represents the input square wave voltage,
while Channel 2 (green) represents the charging and discharging voltage across the capacitor.

The time constant for the theoretical simulation can be found directly using Equation \ref{eq:1} and the component values:

\begin{equation*}
    \begin{aligned}
    \tau = 1 \times 10^3 \cdot 0.1 \times 10^{-6} &= 1 \times 10^{-3} \; \text{s} \\ &= 1.00 \; \text{ms}
    \end{aligned}
\end{equation*}

For the practical circuit, the output observed from the oscilloscope is shown in Figure \ref{fig:5.2}. The same two channels are represented.

\begin{figure}[H]
    \centering
    \includegraphics[width=\linewidth]{ex5_OSC.png}
    \caption{Oscilloscope output for the physical RC circuit driven by a square wave voltage source.}
    \label{fig:5.2}
\end{figure}

From this output, a zoomed-in section of the charging curve can be taken to measure the time constant and fitting
an exponential curve to it, as shown in Figure \ref{fig:5.3}.

\begin{figure}[ht]
    \centering
    \includegraphics[width=\linewidth]{ex5_OSC_tau.png}
    \caption{Zoomed-in section of the charging curve from the oscilloscope output for the physical RC circuit between 0ms and 10ms.}
    \label{fig:5.3}
\end{figure}

Simple curve fitting was performed using Python's SciPy library to fit an exponential curve to the data points.
The model for the exponent fitted to the curve was based on Equation \ref{eq:2}, taking a negative exponent.

The output of the curve gave a time constant \( \tau \) of \( 9.8894 \times 10^{-4} \) s or 0.99 ms for the physical circuit.

Comparing the theoretical and practical time constants gives a percentage error of approximately 1.01\%.

\subsection{Analysis and Discussion} \label{sec:2.4}

The experimental results obtained from the physical RC circuit closely matched the theoretical predictions from the TINA simulation, showing 
a strong agreement. From the calculations made, the theoretical time constant was found to be 1.00 ms, 
while the practical time constant obtained from the oscilloscope data through curve fitting was approximately 0.99 ms, giving
a percentage error of about 1.01\%, which is within the acceptable range for experimental errors.

The close correspondence between the theoretical and practical results indicates that the RC circuit behaves as expected 
within the range of frequencies and voltages used. The exponential charging and discharging curves
observed with the oscilloscope traces confirm the transient response characteristics of RC circuits, with the voltage
dropping to zero as the capacitor remained fully charged before the next cycle.

Small discrepancies between the theoretical and practical results can be attributed to several factors.
These include the parasitic resistances in the lead wires and veroboard connections, which may slightly alter the effective
resistance of the circuit. Additionally, the capacitor and the resistor have tolerances (typically \( \pm \)
10\% for the capacitor, \( \pm \) 5\% for the resistor), which can lead to variations in the actual time constant. Furthermore,
the oscilloscope's measurement accuracy and the resolution of the time base can also introduce minor errors in the trace readings.
Despite these potential sources of error, the overall agreement between theory and experiment indicates high accuracy, making the deviation
negligible for practical purposes.

Overall, the experiment successfully demonstrated the key relationship between resistance, capacitance, and the time-dependent voltage 
behaviour of an RC circuit and validated the theoretical models through practical implementation and measurement.

\subsection{Conclusion} \label{sec:2.5}

The objective of this experiment was to simulate and build a simple RC circuit driven by a square wave voltage source,
observe the voltage across the resistor and capacitor using an oscilloscope, and determine the time constant of the circuit, comparing both
the theoretical and practical values. The time constant otained from the practical circuit (\( \tau \) = 0.99 ms) closely matched the calculated
theoretical prediction (\( \tau \) = 1.00 ms) with a minimal percentage error of approximately 1.01\%. 

This agreement validates the theoretical models for capacitor charging and discharging in an RC circuit, and validates 
Kirchhoff's laws in the energy and charge conservation within electrical circuits. The experiment highlighted the usefulness of
simulation software like TINA for predicting circuit behaviour, as well as the importance of practical measurements using an oscilloscope.

\section{Experiment 7} \label{sec:3}

The objective of this experiment is to simulate and construct an RC-network voltage divider circuit. Using a signal analyser,
the amplitude and phase response of the circuit will be measured over a range of frequencies to plot a Bode plot for computational,
simulated, and achieved results and compared.

\subsection{Theory} \label{sec:3.1}

\subsubsection{Impedance} \label{sec:3.1.1}

"Impedance" can be used in place of "resistance" in order to describe circuits with linear devices, such as resistors, capacitors, and inductors, and
thus generalising Ohm's law. Impedance (Z) can be referred to as the "generalised resistance" and is described by the complex relationship: 
impedance = resistance + reactance, or \( Z = R + jX \). \cite{horowitz1989art}

The generalised Ohm's law can be written as the following \cite{reportguide3Y}:

\begin{equation}\label{eq:3}
    Z = \frac{V}{I}
\end{equation}

The reactance X applies for capacitors and inductors, which are reactive and always 90° out of phase. For resistors, they have reactance (R),
which is always in phase and resistive.

For a capacitor, this reactance (\( X_C \)) is given by \cite{horowitz1989art}:

\begin{equation} \label{eq:4}
    X_C = \frac{1}{\omega C}
\end{equation}

And thus the impedance (\( Z_C \)) of a capacitor is then \cite{horowitz1989art,reportguide3Y}:

\begin{equation}\label{eq:5}
    Z_C = -\frac{j}{\omega C} = \frac{1}{j\omega C}
\end{equation}

with C as the capacitance in farads (F) and \( \omega \) as the angular frequency in hertz (Hz). The complex part of the equation accounts for
the 90° phase shift in the current-voltage curve.

\subsubsection{Voltage Dividers} \label{sec:3.1.2}

Voltage dividers are circuits that produce a fraction of the input voltage as the output voltage for a given voltage input. They are often used
to produce a particular voltage from a larger voltage, either fixed or varying. In general, the division ratio of \( V_{out} \) to \( V_{in} \) is 
not constant as it is dependent on the frequency \( \omega \). \cite{horowitz1989art}

The Thévenin equivalent circuit, which states that "any two-terminal network of resistors and voltage sources is equivalent to a single resistor
R in series with a single voltage source V", can also be generalised with impedance \cite{horowitz1989art,reportguide3Y}:

\begin{equation} \label{eq:6}
    V_{out} = V_{in} \frac{R_2}{R_1 + R_2} = V_{in} \frac{Z_2}{Z_1 + Z_2}
\end{equation}

\subsubsection{Decibels} \label{sec:3.1.3}

To compare the relative amplitude, or magnitudes, of two signals, the logarithmic decibel scale is used. This ratio is typically 
given with the relative intensities \cite{horowitz1989art,reportguide3Y}:

\begin{equation} \label{eq:7}
    dB = 10 \log_{10} \frac{I_2}{I_1}
\end{equation}

However, since the intensity \( \propto \) the amplitude\( ^2 \), and signal amplitudes are most commonly dealt with in circuits, Equation \ref{eq:7}
can be rewritten \cite{reportguide3Y}:

\begin{equation} \label{eq:8}
    \begin{aligned}
        dB  & = 10 \log_{10} \frac{V_2^2}{V_1^2} \\\\
        & = 20 \log_{10} \frac{V_2}{V_1}
    \end{aligned}
\end{equation}

As the voltage V is often the amplitude source of a circuit.

\subsubsection{RC High-Pass Filter} \label{sec:3.1.4}

By combining resistors and capacitors in a circuit it is possible to make frequency-dependent voltage divider with the use of the frequency-dependence
in the impedance of a capacitor \( Z_C = -j / \omega C \). These circuits have the abilities to reject undesired signal frequencies, only allowing
the desired frequencies to pass. \cite{horowitz1989art}

At high frequencies \( \omega \gtrsim 1/RC \), the output is approximately equal to the input and approaches zero at low frequencies.
The -3 dB point at which the curve bends and the capacitor is approximately 63\% charged after a time \( \tau = RC \) may be referred to 
as the "breakpoint" and is given by \cite{horowitz1989art,reportguide3Y}:

\begin{equation} \label{eq:9}
    f_{3\,\text{dB}} = \frac{1}{2 \pi RC}
\end{equation}

At \( \omega = 0 \) the phase shift is the expected +90°. At \( \omega_{3 \, \text{dB}} \) the phase shift changes to +45°, and at
\( \omega = \infty \) the phase shift changes to a flat 0°. \cite{horowitz1989art}

The bandwidth of the circuit is then described as the range over which the response does not drop by over 3 dB; or, the range of frequencies
that can be rejected or passed in a voltage divider circuit. \cite{reportguide3Y}

\subsection{Methodology} \label{sec:3.2}

The circuit was constructed on the TINA software as shown in Figure \ref{fig:7}, which is identical to the circuit in experiment 5 
(§\ref{sec:2.2}) with the difference of a signal analyser in place of an oscilloscope connection. The same resistor (R = 1 k\( \Omega \)) and
capacitor (C = 1 \( \mu \)F) were used. A sinusoidal wave source of 1V was used to drive the circuit.

\begin{figure}[H]
    \centering
    \includegraphics[width=.8\linewidth]{EX7_setup.JPG}
    \caption{Experimental setup constructed on the TINA software.}
    \label{fig:7}
\end{figure}

The same physical circuit as used in experiment 5 (§\ref{sec:2.2}) was used for the practical measurements of this experiment.

\subsection{Results} \label{sec:3.3}

Using the generalised voltage equation (Eq. \ref{eq:6}) and the impedance form for resistance (R, simply) and capacitance (Eq. \ref{eq:5}), the
transfer function for the gain can be approximated:

\begin{equation*}
    \begin{aligned}
        V_{out} & = V_{in} \frac{R}{\frac{1}{j\omega C}+R} \\\\
        & = V_{in} \frac{j\omega RC}{1 + j\omega RC}
    \end{aligned}
\end{equation*}

Such, the ratio of \( V_{out} / V_{in} \) can be found and defined as the complex transfer function \( H(j\omega) \):

\begin{equation*}
    H(j\omega) = \frac{V_{out}}{V_{in}} = \frac{j\omega RC}{1 + j\omega RC}
\end{equation*}

The magnitude of this function can be found by applying Equation \ref{eq:8} directly as it is the ratio of \( V_{out} / V_{in} \).
The phase angle can be determined by taking the arctangent of the imaginary part of the transfer function (1) divided by the real 
part of the transfer function (\( \omega RC \)).

Plotted across a range of frequencies from 1 Hz to 1 MHz (\( 10^0 \; \text{to} \; 10^6 \), as a logarithm), the gain and phase shift of the
RC high-pass voltage divider circuit are found and plotted in Figure \ref{fig:7.1}.

For this plot, the cutoff frequency is found to be approximately \textbf{159 Hz}.

\begin{figure}[H]
    \centering
    \includegraphics[width=\linewidth]{ex7_calculation.png}
    \caption{Bode plot from the calculations of an RC-network voltage divider circuit driven by a sine wave voltage source with \( V_0 \) = 1, C = 1 \( \mu \)F, and R = 1 k\( \Omega \).}
    \label{fig:7.1}
\end{figure}

As the magnitude (gain) was already found on a logarithmic scale (decibels), only frequency is showcased as a logarithmic scale. Despite this,
the gain plot is a log-log plot, whilst the phase angle plot is a log-linear plot with the phase angles linear.

A similar graph was plotted for the TINA simulation results, shown in Figure \ref{fig:7.2} using the Signal Analysis instrument in the
T\&M menu, resulting in the output represented as a Bode plot (gain and phase).

For this plot, the cutoff frequency is found to be approximately \textbf{160 Hz}.

\begin{figure}[ht]
    \centering
    \includegraphics[width=\linewidth]{ex7_TINA.png}
    \caption{Bode plot from the TINA simulation of the RC-network voltage divider circuit driven by a sine wave voltage source.}
    \label{fig:7.2}
\end{figure}

Again, a similar graph was plotted for the results obtained from the physical circuit constructed, shown in Figure \ref{fig:7.3}. Instead of a 
smooth plot, logarithmic values for frequency were chosen (1 Hz, 10 Hz, 100 Hz, 1 kHz, 10 kHz, 100 kHz, 1 MHz) and the amplitude and phase responses 
observed. Results were tabled (\hyperref[sec:A]{Appendix}, table \ref{tab:7}) and graphed, with the magnitude/gain found with Equation \ref{eq:8}.

For this plot, the cutoff frequency is found to be approximately \textbf{542 Hz}.

\begin{figure}[H]
    \centering
    \includegraphics[width=\linewidth]{ex7_circuit.png}
    \caption{Bode plot from the physical RC-network voltage divider circuit driven by a sine wave voltage source.}
    \label{fig:7.3}
\end{figure}

\begin{figure}[ht]
    \centering
    \includegraphics[width=\linewidth]{ex7_comparison.png}
    \caption{Comparison of the gain (top) and phase (bottom) Bode plot curves.}
    \label{fig:7.4}
\end{figure}

For visual comparison of plot agreement, all obtained curves and points were plotted against each other, shown in Figure \ref{fig:7.4}.

The expected cutoff frequency for this circuit can be calculated with Equation \ref{eq:9}, and is determined to be at approximately \textbf{159 Hz}.

\subsection{Analysis and Discussion} \label{sec:3.4}

The results from the computational calculation (Fig. \ref{fig:7.1}), TINA simulation (Fig. \ref{fig:7.2}), and physical circuit (Fig. \ref{fig:7.3})
all show the expected behaviour of an RC high-pass voltage divider. As predicted by the transfer function

\begin{equation*}
    H(j\omega) = \frac{j\omega RC}{1 + j\omega RC}
\end{equation*}

the output amplitude increases with frequency \( \omega \). At low frequencies (\( \omega \ll 1/RC\)), the capacitor impedance dominates
that effectively blocks the input signal and results in a low output voltage. At high frequencies (\( \omega \gtrsim 1/RC \)), the gain approaches
unity (0 dB) and allowing a greater portion of the signal to appear through the resistor, approaching the input voltage. The corresponding phase
response transitions from approximately +90° at low frequencies to 0° at high frequencies, and a +45° phase shift at the cutoff frequency (-3 dB).

The theoretical cutoff frequency, determined with Equation \ref{eq:9}, is found to be approximately \textbf{159 Hz} for R = 1 k\( \Omega \) and 
C = 1 \( \mu \)F. This value very closely matches the value obtained from the TINA simulation of \textbf{160 Hz}, demonstrating a strong agreement
between the analytical and simulated models. However, the experiment cutoff frequency for the physical circuit was measured at approximately
\textbf{542 Hz}, which is an upward deviation of roughly \textbf{241\%} from the calculation.

This discrepancy can be attributed to several experimental factors. The tolerance for a capacitor (often ±10\%) and for a resistor (±5\%) may have
shifted the true RC product, directly affecting the \( f_{3\,\text{dB}} \) as a lower actual capacitance than the expected 1 \(\mu\)F would increase
the cutoff frequency proportionally, similarly were it the resistor. Additionally, parasitic capacitance and resistance in the veroboard and
connecting leads, and internal impedance from the oscilloscope input, can modify the circuit's effective impedance. At higher frequencies, small
parasitic resistances and inductances can distort both amplitude and phase responses.

Despite deviations, the shape and trend of the experiment Bode plots agree with the theoretical expectations (Fig. \ref{fig:7.4}).
The gain plot displays the behaviour expected as discussed, while the phase response plot displays a smooth decrease toward 0° from +90°.
These results confirm that the constructed circuit functions as a high-pass filter voltage divider, allowing greater frequencies to pass while
holding back lower frequencies.

Overall, the experiment successfully illustrated the principles of the frequency-dependent impedance, the relationship between gain and phase
in a reactive circuit, and the analytical calculation of the cutoff frequency expected for such circuit. The minor quantitative discrepancies
can be explained by the realistic non-ideal component behaviours and measurement limitations.

\subsection{Conclusion} \label{sec:3.5}

The aim of this experiment was to investigate the frequency response of an RC high-pass filter voltage divider circuit and 
produce a Bode plot for the values obtained, alongside determining its cutoff frequency, through analytical, simulated, and experimental approaches.
The results obtained for the analytical and TINA simulation were in excellent agreement, both predicting a cutoff frequency near \textbf{159-160 Hz}.

The result obtained from the physical experiment circuit, however, yielded a higher cutoff frequency of approximately \textbf{542 Hz}, a
discrepancy from the other obtained values that may be attributed to component tolerances and parasitic effects within the measurement setup.
Despite this, the overall gain and phase behaviour matched the expected curve form, validating the fundamental relationship between impedance and
frequency in RC circuits.

The experiment results provided an effective demonstration to the usefulness of visualising both amplitude and phase response in a Bode plot,
bridging theoretical results with practical measurements. 

\section{Experiment 12} \label{sec:4}

\subsection{Theory} \label{sec:4.1}

\subsubsection{Diodes}

\subsection{Methodology} \label{sec:4.2}

\begin{figure}[H]
    \centering
    \includegraphics[width=\linewidth]{exp12a_setup.JPG}
    \caption{Experimental setup constructed on the TINA software.}
    \label{fig:12a}
\end{figure}

\begin{figure}[H]
    \centering
    \includegraphics[width=.9\linewidth]{exp12b_setup.JPG}
    \caption{Experimental setup constructed on the TINA software, driven by a sinusoidal waveform.}
    \label{fig:12b}
\end{figure}

\begin{figure}[H]
    \centering
    \includegraphics[width=.9\linewidth]{exp12b_setup_cap.JPG}
    \caption{Experimental setup constructed on the TINA software, with a capacitor in parallel with the resistor and driven by a sinusoidal waveform.}
    \label{fig:12bcap}
\end{figure}

\subsection{Results} \label{sec:4.3}

\begin{figure}[H]
    \centering
    \includegraphics[width=\linewidth]{ex12a_TINA_fb.png}
    \caption{Forward TINA}
    \label{fig:12.1}
\end{figure}

\begin{figure}[H]
    \centering
    \includegraphics[width=\linewidth]{ex12a_circuit_fb.png}
    \caption{Forward circuit}
    \label{fig:12.2}
\end{figure}

\begin{figure}[H]
    \centering
    \includegraphics[width=\linewidth]{ex12a_TINA_rb.png}
    \caption{Reverse TINA}
    \label{fig:12.3}
\end{figure}

\begin{figure}[H]
    \centering
    \includegraphics[width=\linewidth]{ex12a_TINA_rb_zoom.png}
    \caption{Reverse TINA zoom}
    \label{fig:12.4}
\end{figure}

\begin{figure}[H]
    \centering
    \includegraphics[width=\linewidth]{ex12b_TINA_nocap.png}
    \caption{No cap TINA}
    \label{fig:12.5}
\end{figure}

\begin{figure}[H]
    \centering
    \includegraphics[width=\linewidth]{ex12b_circuit_nocap.png}
    \caption{No cap circuit}
    \label{fig:12.6}
\end{figure}

\begin{figure}[H]
    \centering
    \includegraphics[width=\linewidth]{ex12b_TINA_cap.png}
    \caption{Cap TINA}
    \label{fig:12.7}
\end{figure}

\begin{figure}[H]
    \centering
    \includegraphics[width=\linewidth]{ex12b_circuit_cap.png}
    \caption{Cap circuit}
    \label{fig:12.8}
\end{figure}

\subsection{Analysis and Discussion} \label{sec:4.4}


\subsection{Conclusion} \label{sec:4.5}



\section{Experiment 19} \label{sec:5}

\section{Experiment 20} \label{sec:6}

\section{Experiment 21} \label{sec:7}

\section{Experiment 25} \label{sec:8}

\section{Experiment 27} \label{sec:9}

% References - switch to single column
\onecolumn

\bibliographystyle{IEEEtran}
\bibliography{PHYC30170References} \label{sec:ref}

% Appendix - single column
\section*{Appendix} \label{sec:A}
\addcontentsline{toc}{section}{Appendix}

\begin{table}[H]
    \centering
    \caption{Table for the values obtained for the physical experimental circuit.}
    \vspace{1em}
    \label{tab:7}
    \begin{tabular}{|
    >{\columncolor[HTML]{EFEFEF}}c |c|c|c|}
    \hline
    \cellcolor[HTML]{C0C0C0}f (Hz) & \cellcolor[HTML]{C0C0C0}$V_{in}$ (V) & \cellcolor[HTML]{C0C0C0}$V_{out}$ & \cellcolor[HTML]{C0C0C0}$\phi$ (°) \\ \hline
    1    & 8.52 & 61.1m & 87    \\ \hline
    10   & 1.01 & 61.2m & 84.8  \\ \hline
    100  & 1.02 & 502m  & 59.68 \\ \hline
    1k   & 970m & 1.00  & 6.95  \\ \hline
    10k  & 950m & 1.02  & 0.14  \\ \hline
    100k & 970m & 1.02  & 0.56  \\ \hline
    1M   & 930m & 970m  & 0.90  \\ \hline
    \end{tabular}%
\end{table}

\end{document}