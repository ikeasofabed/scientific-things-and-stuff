\documentclass[11pt, a4paper, twocolumn]{article}

% Fonts and math (Elsevier-like Times style)
\usepackage{graphicx} % Required for inserting images
\usepackage{array} % For extra column formatting
\usepackage{amsmath, amssymb, cancel} % for equation environment
\usepackage{float,geometry}
\geometry{margin=2cm}
\usepackage[skip=10pt]{parskip}
\usepackage{setspace}
\usepackage{hyperref}
\usepackage{cite, bookmark}
\usepackage{url}
\usepackage{xcolor}
\usepackage{listings}

% --- Abstract styling (elsarticle look)
\usepackage{abstract}
\renewcommand{\abstractnamefont}{\normalfont\bfseries}
\renewcommand{\abstracttextfont}{\normalfont}
\renewenvironment{abstract}{%
  \par\noindent\rule{\linewidth}{0.4pt}\par
  \begin{center}\bfseries Abstract\end{center}%
}{%
  \par\rule{\linewidth}{0.4pt}\par
}

% --- Section styling (similar to elsarticle
\usepackage{titlesec}
\titleformat{\section}{\normalfont\large\bfseries}{\thesection.}{0.5em}{}
\titleformat{\subsection}{\normalfont\normalsize\bfseries}{\thesubsection.}{0.5em}{}

% Table of contents styling
\usepackage{tocloft}
\cftsetindents{section}{0em}{2em}
\cftsetindents{subsection}{1em}{2em}
\renewcommand\cfttoctitlefont{\hfill\Large\bfseries}
\renewcommand\cftaftertoctitle{\hfill\mbox{}}

\graphicspath{ {./images/} }

\definecolor{blurple}{HTML}{5865F2}
\definecolor{backcolour}{HTML}{272823}

\hypersetup{
    colorlinks=true,
    linkcolor=black,
    urlcolor=black,
    citecolor=blurple,
}

\urlstyle{same}

\DeclareMathOperator{\sinc}{sinc}

\pagenumbering{arabic}
\renewcommand{\arraystretch}{1.4}

% Python code export
\lstset{
    language=Python,
    basicstyle=\ttfamily\scriptsize,
    keywordstyle=\color{teal},       
    commentstyle=\color{gray},        
    stringstyle=\color{magenta},  
    showstringspaces=false,
    frame=single,
    breaklines=true,
    numbers=left,
    numberstyle=\tiny\color{gray}
}

\setcounter{secnumdepth}{5}
\setcounter{tocdepth}{5}

%%%%%%%%%%%%%%%%%%%%%%%%%%%%%%%%%%%

\title{%

\includegraphics[width=0.2\linewidth]{ucd_brandmark_black.jpg} \vspace{0.5cm}\\

{\Large\bfseries UCD School of Physics}\vspace{1.5cm}\\

{\large PHYC30170 Physics Astronomy and Space Lab I}\\
{\bfseries Electronics Laboratory}

}
\author{Joana C.C. Adao \\ \small Student No.: 23311051}
\date{18 October 2025}

\begin{document}

% Title page - single column
\onecolumn
\maketitle

% Abstract - single column
\thispagestyle{empty}
\begin{abstract}

This is the abstract.

\end{abstract}

\newpage

% Table of Contents - single column
\tableofcontents

\newpage

%%%%%%%%%%%%%%%%%%%%%%%%%%%%%%%%%%%

% Start main content in two columns
\twocolumn

\section{Introduction} \label{sec:1}

\section{Experiment 5} \label{sec:2}

The objective of this experiment is to simulate and build a simple RC circuit driven by a square wave voltage source.
The voltage across the resistor and capacitor will be observed using an oscilloscope, and the time constant of the circuit will be determined.

\subsection{Theory} \label{sec:2.1}

\subsubsection{Kirchhoff's Laws} \label{sec:2.1.1}

Kirchhoff has two primary and distinct laws that deal with the conservation of charge and energy within electrical circuits. These are:

\textbf{Kirchhoff's Current Law (KCL):} This law states that the sum of currents entering a junction (or node) in an electrical circuit 
must equal the sum of currents leaving that junction. Essentially, it is a statement of the conservation of electric charge.

\textbf{Kirchhoff's Voltage Law (KVL):} This law states that the sum of electrical charge, or the voltage drop,
around any closed loop is zero. Essentially, it is a statement of the conservation of energy within an electrical circuit. \cite{horowitz1989art}

\subsubsection{Time Constant} \label{sec:2.1.2}

The time constant, denoted by the Greek letter tau (\( \tau \)), is a measure of the time it takes for a system to respond to a change in input. 
For electrical circuits, the time constant is often used to describe the behavior of an RC (resistor-capacitor) circuit.

The time constant for an RC circuit is given by the formula:

\begin{equation} \label{eq:1}
    \tau = R C
\end{equation}

And such, the solution to the equation for a circuit's voltage response over time is then given by:

\begin{equation} \label{eq:2}
    V(t) = V_f + Ae^{-\frac{t}{\tau}}
\end{equation}

where R is the resistance in ohms (\( \Omega \)), C is the capacitance in farads (F), \( V(t) \) is the voltage at time t,
\( V_f \) is the final voltage, A is a constant determined by initial conditions. \cite{horowitz1989art}

As the capacitor charges, the slope (which is proportional to the current) is proportional to the remaining voltage. Therefore, the
waveform produced is exponential in nature. For \( t \gg \tau \), the voltage across the capacitor approaches its final value \( V_f \) asymptotically. \cite{horowitz1989art}

\subsection{Methodology} \label{sec:2.2}

The circuit was constructed on the TINA software as shown in Figure \ref{fig:5}. 
A resistor of 1 k\( \Omega \) and a capacitor of 0.1 \( \mu \)F were connected in series in the circuit.
A square wave voltage source was used to drive the circuit, with a frequency of 1 kHz and an amplitude of 5 V.
The voltage across the capacitor and resistor was observed using an oscilloscope.

\begin{figure}[H]
    \centering
    \includegraphics[width=.9\linewidth]{ex5_setup.JPG}
    \caption{Experimental setup constructed on the TINA software.}
    \label{fig:5}
\end{figure}

A similar physical circuit was then constructed on a veroboard using the same components as in the simulation.
The oscilloscope was used to supply the square wave voltage source and to observe the voltage across the capacitor and resistor.

\subsection{Results} \label{sec:2.3}

Results were gathered from both the TINA simulation and the physical circuit using the oscilloscope. The graphs produced are shown in
Figures \ref{fig:5.1} and \ref{fig:5.2}.

\begin{figure}[ht]
    \centering
    \includegraphics[width=\linewidth]{ex5_TINA.png}
    \caption{Oscilloscope output from the TINA simulation driven by a square wave voltage source.}
    \label{fig:5.1}
\end{figure}

The oscilloscope shows the two channels: Channel 1 (blue) represents the input square wave voltage,
while Channel 2 (green) represents the charging and discharging voltage across the capacitor.

The time constant for the theoretical simulation can be found directly using Equation \ref{eq:1} and the component values:

\begin{equation*}
    \begin{aligned}
    \tau = 1 \times 10^3 \cdot 0.1 \times 10^{-6} &= 1 \times 10^{-3} \; \text{s} \\ &= 1.00 \; \text{ms}
    \end{aligned}
\end{equation*}

For the practical circuit, the output observed from the oscilloscope is shown in Figure \ref{fig:5.2}. The same two channels are represented.

\begin{figure}[H]
    \centering
    \includegraphics[width=\linewidth]{ex5_OSC.png}
    \caption{Oscilloscope output for the physical RC circuit driven by a square wave voltage source.}
    \label{fig:5.2}
\end{figure}

From this output, a zoomed-in section of the charging curve can be taken to measure the time constant and fitting
an exponential curve to it, as shown in Figure \ref{fig:5.3}.

\begin{figure}[ht]
    \centering
    \includegraphics[width=\linewidth]{ex5_OSC_tau.png}
    \caption{Zoomed-in section of the charging curve from the oscilloscope output for the physical RC circuit between 0ms and 10ms.}
    \label{fig:5.3}
\end{figure}

Simple curve fitting was performed using Python's SciPy library to fit an exponential curve to the data points.
The model for the exponent fitted to the curve was based on Equation \ref{eq:2}, taking a negative exponent.

The output of the curve gave a time constant \( \tau \) of \( 9.8894 \times 10^{-4} \) s or 0.99 ms for the physical circuit.

Comparing the theoretical and practical time constants gives a percentage error of approximately 1.01\%.

\subsection{Analysis and Discussion} \label{sec:2.4}

The experimental results obtained from the physical RC circuit closely matched the theoretical predictions from the TINA simulation, showing 
a strong agreement. From the calculations made, the theoretical time constant was found to be 1.00 ms, 
while the practical time constant obtained from the oscilloscope data through curve fitting was approximately 0.99 ms, giving
a percentage error of about 1.01\%, which is within the acceptable range for experimental errors.

The close correspondence between the theoretical and practical results indicates that the RC circuit behaves as expected 
within the range of frequencies and voltages used. The exponential charging and discharging curves
observed with the oscilloscope traces confirm the transient response characteristics of RC circuits, with the voltage
dropping to zero as the capacitor remained fully charged before the next cycle.

Small discrepancies between the theoretical and practical results can be attributed to several factors.
These include the parasitic resistances in the lead wires and veroboard connections, which may slightly alter the effective
resistance of the circuit. Additionally, the capacitor and the resistor have tolerances (typically \( \pm \)
10\% for the capacitor, \( \pm \) 5\% for the resistor), which can lead to variations in the actual time constant. Furthermore,
the oscilloscope's measurement accuracy and the resolution of the time base can also introduce minor errors in the trace readings.
Despite these potential sources of error, the overall agreement between theory and experiment indicates high accuracy, making the deviation
negligible for practical purposes.

Overall, the experiment successfully demonstrated the key relationship between resistance, capacitance, and the time-dependent voltage 
behaviour of an RC circuit and validated the theoretical models through practical implementation and measurement.

\subsection{Conclusion} \label{sec:2.5}

The objective of this experiment was to simulate and build a simple RC circuit driven by a square wave voltage source,
observe the voltage across the resistor and capacitor using an oscilloscope, and determine the time constant of the circuit, comparing both
the theoretical and practical values. The time constant otained from the practical circuit (\( \tau \) = 0.99 ms) closely matched the calculated
theoretical prediction (\( \tau \) = 1.00 ms) with a minimal percentage error of approximately 1.01\%. 

This agreement validates the theoretical models for capacitor charging and discharging in an RC circuit, and validates 
Kirchhoff's laws in the energy and charge conservation within electrical circuits. The experiment highlighted the usefulness of
simulation software like TINA for predicting circuit behaviour, as well as the importance of practical measurements using an oscilloscope.

\section{Experiment 7} \label{sec:3}

The objective of this experiment is to simulate and construct an RC-network voltage divider circuit. Using a signal analyser,
the amplitude and phase response of the circuit will be measured over a range of frequencies to plot a Bode plot.

\subsection{Theory} \label{sec:3.1}


\subsection{Methodology} \label{sec:3.2}



\subsection{Results} \label{sec:3.3}

\begin{figure}[H]
    \centering
    \includegraphics[width=\linewidth]{ex7_TINA.png}
    \caption{Bode plot from the TINA simulation of the RC-network voltage divider circuit driven by a sine wave voltage source.}
    \label{fig:7.1}
\end{figure}


\begin{figure}[H]
    \centering
    \includegraphics[width=\linewidth]{ex7_circuit.png}
    \caption{Bode plot from the physical RC-network voltage divider circuit driven by a sine wave voltage source.}
    \label{fig:7.2}
\end{figure}

\subsection{Analysis and Discussion} \label{sec:3.4}


\subsection{Conclusion} \label{sec:3.5}

\section{Experiment 12} \label{sec:4}

\section{Experiment 19} \label{sec:5}

\section{Experiment 20} \label{sec:6}

\section{Experiment 21} \label{sec:7}

\section{Experiment 25} \label{sec:8}

\section{Experiment 27} \label{sec:9}

% References - switch to single column
\onecolumn

\bibliographystyle{IEEEtran}
\bibliography{PHYC30170References} \label{sec:ref}

% Appendix - single column
\section*{Appendix} \label{sec:A}
\addcontentsline{toc}{section}{Appendix}


% Your appendix content here

\end{document}