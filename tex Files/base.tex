\documentclass{article}
\usepackage[english]{babel} % To obtain English text with the blindtext package
\usepackage{blindtext}
\usepackage{graphicx} % Required for inserting images
\usepackage{array} % For extra column formatting
\usepackage{amsmath} %for equation environment
\usepackage{float}
\usepackage{parskip} % For gaps between para
\usepackage{setspace}
\usepackage{pdfpages}
\usepackage{abstract}
\usepackage[export]{adjustbox}

\graphicspath{ {./images/} }

\title{Experiment Number: Experiment Name}
\author{Your Name}
\date{\today} % shows today's date


\begin{document}

\begin{titlepage}
    \centering % centres text

    {\Huge\bfseries UCD School of Physics}\\
    {\medskip}
    {\Large Module Number and Title}\\
    {\Large Experiment Number: Experiment Name}
   {\bigskip}
   
   {\Large \textbf{Date: 23 October 2024 }}
   
   {\Large \textbf{Name:} Your Name}
   
   {\Large \textbf{Lab Partners' Names:} Name 1, Name 2}
    \vspace*{2cm}

\end{titlepage}

``This puts the paragraph in quotes. 
\LaTeX{}, \TeX{} are the two ways to write Latex.''

\par You can use a slash command to split the paragraphs and put \par everything \par 
in \par its \par own \par paragragh.

\section*{A subsection flush left.}

\begin{flushleft}
This is the standard view of pargraph layout, with words not centred but actually
coming from the left.
\end{flushleft}

\section*{A subsection flush right.}

\begin{flushright}
The same can be done for the right side of the document when writing paragraphs, the
same way you'd be able to do it with any writing program.
\end{flushright}

\setlength{\parindent}{20pt}

\section*{This is a section}
\textbf{First paragraph} of a section which, as you can see, is not indented. 
This is more text in the paragraph. This is more text in the paragraph.

\textbf{Second paragraph}. As you can see it is indented. 
This is more text in the paragraph. This is more text in the paragraph. 

\noindent\textbf{Third paragraph}. This too is not indented due to use 
of \texttt{\string\noindent}. This is more text in the paragraph. 
This is more text in the paragraph.  
The current value of \verb|\parindent| is \the\parindent. 
This is more text in the paragraph.

\indent you can also indent \indent manually

\noindent A new paragraph with some text, then an \verb|\indent|\indent command. Next, some inline math which also has an indent $y\indent x$. \verb|\indent| also works when used in an \verb|\hbox| such as \verb|\hbox{A\indent B}| which produces \hbox{A\indent B}.

\section*{Bold, Italics and Underline}

This \textit{italicises} the text.
This \textbf{boldens} the text.
This \underline{underlines} the text.

\par
\emph{Emphasis} is not the same as the others.
\textit{In a fully italicised text \emph{emphasis} becomes normal.}
\textbf{And in a fully bold body \emph{emphasis} is italic bold.} It depends on the context.

\section*{Lists}

Lists are easy to create:
\begin{itemize}
  \item List entries start with the \verb|\item| command.
  \item Individual entries are indicated with a black dot, a so-called bullet.
  \item The text in the entries may be of any length.
\end{itemize}

Numbered (ordered) lists are easy to create:
\begin{enumerate}
  \item Items are numbered automatically.
  \item The numbers start at 1 with each use of the \texttt{enumerate} environment.
  \item Another entry in the list
\end{enumerate}

\begin{description}
    \item This is an entry \textit{without} a label.
    \item[Something short] A short one-line description.
    \item[Something long] A much longer description. I'm not using the blindtext package
    because VSCode seems to not like it. So I am just typing random things and hoping
    the difference in using ``description'' is clearly visible, which is only really
    possible when you have a decently sized piece of text, hence why I am writing random
    nonsense. This is just a showcase, here are even some more words and I'll add some
    more while I am here and at it. It is very cold today because there will be a storm
    tomorrow, it is provisioned to be very windy.
 \end{description}


Change the labels using \verb|\item[label text]| in an \texttt{itemize} environment
\begin{itemize}
  \item This is my first point
  \item Another point I want to make 
  \item[!] A point to exclaim something!
  \item[NOTE] This entry has no bullet
  \item[] A blank label?
\end{itemize}

\vspace{10pt}

Change the labels using \verb|\item[label text]| in an \texttt{enumerate} environment
\begin{enumerate}
  \item This is my first point
  \item Another point I want to make 
  \item[!] A point to exclaim something!
  \item[NOTE] This entry has no bullet
  \item[] A blank label?
\end{enumerate}

\section*{Nested Lists}

\begin{enumerate}
    \item The labels consists of sequential numbers
    \begin{itemize}
      \item The individual entries are indicated with a black dot, a so-called bullet
      \item The text in the entries may be of any length
      \begin{description}
      \item[Note:] I would like to describe something here
      \item[Caveat!] And give a warning here
      \end{description}
    \end{itemize}
    \item The numbers starts at 1 with each use of the \texttt{enumerate} environment
 \end{enumerate}

\begin{enumerate}
   \item First level item
   \item First level item
   \begin{enumerate}
     \item Second level item
     \item Second level item
     \begin{enumerate}
       \item Third level item
       \item Third level item
       \begin{enumerate}
         \item Fourth level item
         \item Fourth level item
       \end{enumerate}
     \end{enumerate}
   \end{enumerate}
 \end{enumerate}

\par

\begin{itemize}
   \item First level item
   \item First level item
   \begin{itemize}
     \item Second level item
     \item Second level item
     \begin{itemize}
       \item Third level item
       \item Third level item
       \begin{itemize}
         \item Fourth level item
         \item Fourth level item
       \end{itemize}
     \end{itemize}
   \end{itemize}
 \end{itemize}


\begin{figure}[ht]
    \includegraphics[width=\textwidth,left]{UCDLogo.png}
    \caption{UCD Logo Image}
    \label{fig:ucdlogo}
\end{figure}

\listoffigures

\section*{Tables}

\begin{center}
    \begin{tabular}{ c c c }
     cell1 & cell2 & cell3 \\ 
     cell4 & cell5 & cell6 \\  
     cell7 & cell8 & cell9    
    \end{tabular}
\end{center}

\begin{center}
    \begin{tabular}{ |c|c|c| } 
     \hline
     cell1 & cell2 & cell3 \\ 
     cell4 & cell5 & cell6 \\ 
     cell7 & cell8 & cell9 \\ 
     \hline
    \end{tabular}
\end{center}

\begin{center}
    \begin{tabular}{||c c c c||} 
     \hline
     Col1 & Col2 & Col2 & Col3 \\ [0.5ex] 
     \hline\hline
     1 & 6 & 87837 & 787 \\ 
     \hline
     2 & 7 & 78 & 5415 \\
     \hline
     3 & 545 & 778 & 7507 \\
     \hline
     4 & 545 & 18744 & 7560 \\
     \hline
     5 & 88 & 788 & 6344 \\ [1ex] 
     \hline
    \end{tabular}
\end{center}


\begin{center}
    \begin{tabular}{ | m{5em} | m{1cm}| m{1cm} | } 
    \hline
    cell1 dummy text dummy text dummy text& cell2 & cell3 \\ 
    \hline
    cell1 dummy text dummy text dummy text & cell5 & cell6 \\ 
    \hline
    cell7 & cell8 & cell9 \\ 
    \hline
    \end{tabular}
\end{center}

\end{document}
