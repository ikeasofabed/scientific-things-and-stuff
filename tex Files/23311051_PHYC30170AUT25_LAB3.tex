\documentclass[11pt, a4paper, twocolumn]{article}

% Fonts and math (Elsevier-like Times style)
\usepackage{graphicx} % Required for inserting images
\usepackage{array} % For extra column formatting
\usepackage{amsmath, amssymb, cancel} % for equation environment
\usepackage{float,geometry}
\geometry{margin=2cm}
\usepackage[skip=10pt]{parskip}
\usepackage{setspace}
\usepackage{hyperref}
\usepackage{cite, bookmark}
\usepackage{url}
\usepackage[table,xcdraw]{xcolor}
\usepackage[export]{adjustbox}
\usepackage{listings,pdfpages}
\usepackage{dblfloatfix}

% --- Abstract styling (elsarticle look)
\usepackage{abstract}
\renewcommand{\abstractnamefont}{\normalfont\bfseries}
\renewcommand{\abstracttextfont}{\normalfont}
\renewenvironment{abstract}{%
  \par\noindent\rule{\linewidth}{0.4pt}\par
  \begin{center}\bfseries Abstract\end{center}%
}{%
  \par\rule{\linewidth}{0.4pt}\par
}

% --- Section styling (similar to elsarticle
\usepackage{titlesec}
\titleformat{\section}{\normalfont\large\bfseries}{\thesection.}{0.5em}{}
\titleformat{\subsection}{\normalfont\normalsize\bfseries}{\thesubsection.}{0.5em}{}

% Table of contents styling
\usepackage{tocloft}
\cftsetindents{section}{0em}{2em}
\cftsetindents{subsection}{1em}{2em}
\renewcommand\cfttoctitlefont{\hfill\Large\bfseries}
\renewcommand\cftaftertoctitle{\hfill\mbox{}}
\renewcommand\cftfigfont{\footnotesize}
\renewcommand\cftfigpagefont{\footnotesize}
\renewcommand\cftloftitlefont{\large}

\graphicspath{ {./images/} }

\definecolor{blurple}{HTML}{5865F2}
\definecolor{backcolour}{HTML}{272823}

\hypersetup{
    colorlinks=true,
    linkcolor=black,
    urlcolor=black,
    citecolor=blurple,
}

\urlstyle{same}

\DeclareMathOperator{\sinc}{sinc}

\pagenumbering{arabic}
\renewcommand{\arraystretch}{1.4}

% Python code export
\lstset{
    language=Python,
    basicstyle=\ttfamily\scriptsize,
    keywordstyle=\color{teal},       
    commentstyle=\color{gray},        
    stringstyle=\color{magenta},  
    showstringspaces=false,
    frame=single,
    breaklines=true,
    numbers=left,
    numberstyle=\tiny\color{gray}
}

\setcounter{secnumdepth}{5}
\setcounter{tocdepth}{5}

%%%%%%%%%%%%%%%%%%%%%%%%%%%%%%%%%%%

\title{%

\includegraphics[width=0.2\linewidth]{ucd_brandmark_black.jpg} \vspace{0.5cm}\\

{\Large\bfseries UCD School of Physics}\vspace{1.5cm}\\

{\large PHYC30170 Physics Astronomy and Space Lab I}\\
{\bfseries Modulus of Rigidity}

}
\author{Joana C.C. Adao \\ \small Student No.: 23311051}
\date{5 November 2025}

\begin{document}

% Title page - single column
\onecolumn
\maketitle

% Abstract - single column
\thispagestyle{empty}
\begin{abstract}

This experiment sought to investigate the torsional oscillatory response for three metal specimen rods: brass, stainless steel, and aluminium. The
experiment was conducted over a temperature range from 60°C to 180°C, investigating effects on the modulus of rigidity and internal friction.
Amplitude-frequency data collected with a laser sensor and National Instruments Programme were fitted with a Lorentzian curve to identify resonant
frequency peaks and bandwidths. From these, the shear modulus and internal friction for all three metals was calculated as functions of temperature.
All three specimen showcased a gradual decrease in resonant frequency and modulus of rigidity with increasing temperatures, consistent with theoretical
predictions. Stainless steel displayed the lowest internal friction and sharpest resonant peaks, while aluminium displayed broader
peaks and greater damping. Although experimental values were significantly lower than expected values from published literature, the physical trends
aligned well with theoretical expectations, highlighting the temperature dependence of the shear modulus and energy dissipation in metals.

\end{abstract}

\newpage

% Table of Contents - single column
\setcounter{page}{1}
\tableofcontents

\newpage

%%%%%%%%%%%%%%%%%%%%%%%%%%%%%%%%%%%

% Start main content in two columns
\twocolumn

\section{Theory} \label{sec:2}

\subsection{Torsional Oscillator} \label{sec:2.1}

The torsional oscillator, or torsional pendulum, consists of a rigid mass hanging by a thin wire or rod that resists twisting with a resisting
torque, which acts to restore the wire to its original state. The rigid body undergoes angular oscillations due to this torque, and for
relatively small angles of twisting the magnitude of the restoring torque (\( \tau \)) will be directly proportional to the angle \cite{fitzpatrick2006torsion}:

\begin{equation}
    \tau = -C \theta
\end{equation}

with \( C \) as the torsional constant, which is dependent on the shear modulus, or modulus of rigidity, of the wire's material, and can be 
expressed as \cite{shanker1985temperature}: 

\begin{equation} \label{eq:2}
    C = \frac{\pi G r^4}{2l}
\end{equation}

with \( r\) as the radius and \( l\) length of the rod. For a body of moment of inertia \( I \) the rotational equation of motion of 
the system can be written as \cite{shanker1985temperature,fitzpatrick2006torsion,tatum_classical_mechanics}:

\begin{equation}
    I \frac{d^2 \theta}{dt^2} = -C \theta
\end{equation}

For the undamped case, the solution is simply \( \theta = \theta_0 \cos (\omega t - \phi) \), with \( \omega = \sqrt{C/I} \). For a damped
oscillator, a damping factor \( \gamma \) is considered in the propagation of oscillations \cite{shanker1985temperature,filippini2020torsional_oscillator}:

\begin{equation}
    I \frac{d^2 \theta}{dt^2} + \mathbf{\gamma \frac{d \theta}{dt}} + C \theta = 0
\end{equation}

This gives the result \( T_0 \exp(i\omega t) \), with \( T_0 \) as the period of oscillation \( 2 \pi \sqrt{I/C} \).

\subsection{Modulus of Rigidity} \label{sec:2.2}

The modulus of rigidity, or shear modulus, is a measure of a material's resistance to shear stress deformation. It is expressed as the 
ration of shear stress \( \tau \) to shear strain \( \gamma \) \cite{study_modulus_of_rigidity_2023,samaterials_shear_modulus_2025}:

\begin{equation} \label{eq:5}
    G = \frac{\tau}{\gamma}
\end{equation}

with \( G \) as the modulus of rigidity. It qualifies how rigid or stiff the material is under shear loading, such that a higher value of G 
indicates a material harder to deform under shear stress. \cite{samaterials_shear_modulus_2025}

Knowing that for low damping \( \theta_0 \) will be at maximum when \( \omega \approx \sqrt{C/I} = \omega_0 \) C can be substituted from 
Equation \ref{eq:2} into Equation \ref{eq:5} to give \cite{shanker1985temperature}:

\begin{equation} \label{eq:6}
    G = \frac{2 I l \omega_0^2}{\pi r^4}
\end{equation}

The modulus of rigidity is influence by factors such as temperature, confining pressure, microstructure of the inner particles, material
composition, and grain size. Numerical investigations of granular packings show that the modulus of rigidity is sensitive to the coordination
number of contacts and load magnitudes, with poorly coordinated packings having a shear modulus that varies proportionally. 
\cite{samaterials_shear_modulus_2025,agnolin2007internal}

\subsection{Internal Friction} \label{sec:2.3}

Internal friction in a material refers to the dissipation of mechanical energy, or the force-resisting motion between elements,
as it undergoes "elastic" or oscillatory deformation.
\cite{cuni_internal_friction_2025,HYDE2015211}

This internal friction \( Q^{-1} \) can be taken as \cite{shanker1985temperature}:

\begin{equation} \label{eq:7}
    Q^{-1} \equiv \frac{\gamma \omega_0}{C} = \frac{\Delta \omega}{\sqrt{3} \omega_0}
\end{equation}

with \( \Delta \omega \) as the full width of the response curve at half the maximum amplitude \( \sqrt{3} \gamma \omega_0^2 / C \). This
relation shows that internal friction increases proportionally with resonance bandwidth \( \Delta \omega \) increases, such that the internal
loss is greater and the resonance is poorer. \cite{shanker1985temperature}

\section{Methodology}

The experiment was set up as shown in Figure \ref{fig:1}, with a laser pointing to a laser sensor by reflecting off of the mirror attached
to the magnet attached to the specimen rod. The magnet was positioned between the Helmholtz coils, ensuring that the specimen rod was through
the heater coil and clamped securely to the support stand.

\begin{figure}[ht]
    \centering
    \includegraphics[width=.8\linewidth]{modrig_diagram.png}
    \caption{Diagram of the experimental setup: 1, Specimen rod; 2, heater coil; 3, magnet; 4, mirror; 5, clamp support; 6, support stand; 7, Helmholtz coil.}
    \label{fig:1}
\end{figure}

With light taps to generate perturbations, the rod's vibrating capacity is checked to ensure proper readings of the amplitude as it vibrates 
with the torsional oscillations generated by varying frequencies with an oscilloscope, fed through the Helmholtz coil.

The laser sensor produces readings fed through a DAC that can be plotted and graphed to determine the modulus of rigidity and internal 
friction of three metals: brass, stainless steel, and aluminium.

The data collected is then plotted with Python code to generate graphs that visually represent the data and relationships.

\subsection{National Instrument Setup}

The National Instruments Programme (NIP) device must be set up for the multiple readings to be converted into computational data. This
instrument is connected to the apparatus and a computer and provided voltages determined with simple coding, and interprets the laser sensor
DAC values.

The voltage provided by the NIP must be converted into frequencies for graphical readings. This can be achieved by calibrating the instrument.
For varying voltage readings, the corresponding frequencies read from the oscilloscope were noted on a computer. A linear relationship is
the plotted which determines the ratio of voltage to frequency.

\subsection{Health and Safety}

Due to the presence of a high-intensity laser, care must be taken to avoid direct eye contact. The heater coil is very hot to the touch as
are the metal specimen rods, so care must be taken when removing the metal rods from the apparatus as to not burn and injure oneself.

\section{Results}

From the Python code, graphs of the amplitude obtained from the laser sensor plotted against the frequency (voltage) from the oscilloscope were
used to find the resonant frequency of each metal at varying temperatures 60°C, 90°C, 120°C, 150°C, and 180°C. With these values, the modulus
of rigidity was found and plotted alongside the internal friction for each metal type: brass, stainless steel, and aluminium.

\subsection{Amplitude vs. Frequency}

The amplitude-frequency plots for the brass, stainless steel, and aluminium rods were plotted with a Lorentzian curve fitted, using:

\begin{equation}
    L(x) = y_0 + \frac{2A}{\pi} \frac{\omega_0}{4 (x-x_c)^2 + \omega ^2}
\end{equation}

with \( A \) as the area under the curve, \( \omega_0 \) as the width of the curve at half maximum, \( y_0 \) the initial value, and \( x_c \) the 
estimated value of the peak. From this, the resonant peak was found from the plotted data fit.

\subsubsection{Brass}

\begin{figure*}[ht]
    \centering
    \includegraphics[width=.96\textwidth]{allbrass.png}
    \caption{Brass values for varying temperatures with fitted Lorentzian.}
    \label{fig:b2}
\end{figure*}

\begin{figure*}[!b]
    \centering
    \includegraphics[width=.96\textwidth]{allstainlesssteel.png}
    \caption{Stainless steel values for varying temperatures with fitted Lorentzian.}
    \label{fig:ss2}
\end{figure*}

For the brass rod, the different obtained data points and respective Lorentzian curve fits are shown with Figure \ref{fig:b2}. In Figure \ref{fig:b1},
a decreasing trend in frequency can be seen with increasing temperature, each peak remaining at approximately the same amplitude and therefore 
intensity. The curves for this metal type were narrow and symmetric and showed a clear-temperature dependence for the resonant frequency peak.
The full width at half maximum \( omega_0 \) remained relatively constant, which indicates that the damping on the material was largely uniform
over the entire temperature range. The resonant frequency over the range of temperatures was approximately \textbf{194.53 Hz}.

\begin{figure}[H]
    \centering
    \includegraphics[width=.775\linewidth]{brass_val.png}
    \caption{Plotted lines for brass values.}
    \label{fig:b1}
\end{figure}

\subsubsection{Stainless Steel}

\begin{figure}[!b]
    \centering
    \includegraphics[width=.775\linewidth]{stainlesssteel_val.png}
    \caption{Plotted lines for stainless steel values.}
    \label{fig:ss1}
\end{figure}

\begin{figure*}[!t]
    \centering
    \includegraphics[width=.96\textwidth]{allalluminium.png}
    \caption{Aluminium values for varying temperatures with fitted Lorentzian.}
    \label{fig:a2}
\end{figure*}

For the stainless steel rod, the data points obtained from the equipment were plotted with a Lorentzian curve fit for each temperature,
shown in Figure \ref{fig:ss2}. In Figure \ref{fig:ss1} showed a similar trend in decreasing resonant frequency with increasing temperature.
The resonance peaks were very sharply defined and tall, which indicates high oscillation quality and low energy dissipation in the material.
The variations in amplitude were very similar, with only minimal variations, making the stainless steel rod the most stable in terms of the 
resonant frequency and amplitude response. According to papers and lab coordinator however, the resonant frequency for this metal should be much
higher than what was obtained, and should be approximately between what was obtained for brass and aluminium.
The average resonant frequency for this metal was approximately \textbf{92.21 Hz}.

\subsubsection{Aluminium}

For aluminium, the data obtained was plotted on a graph and a Lorentzian curve was fitted to what was obtained, shown in Figure \ref{fig:a2}. The
points as plotted in Figure \ref{fig:a1} show a wider curve and considerably smaller resonance peaks. The graph shows a similar trend in decreasing
resonant frequency with increasing temperature, but the amplitude is seen to be affected alongside it with an exponential decrease with linearly
increasing temperatures. At higher temperatures, the aluminium rod appears to begin to flatten. The broader peaks suggest a greater damping effect
and a decreased ability of the material to maintain torsional oscillations at resonance. The resonant frequency for aluminium averaged over
the temperature range is approximately \textbf{178.59 Hz}.

\begin{figure}[ht]
    \centering
    \includegraphics[width=.775\linewidth]{aluminium_val.png}
    \caption{Plotted lines for aluminium values.}
    \label{fig:a1}
\end{figure}

\subsection{Modulus of Rigidity}

The experimentally obtained values for the modulus of rigidity are shown in Table \ref{tab:modrig} and shown graphically in Figure \ref{fig:modr}
for brass, stainless steel, and aluminium with respect to temperature. Across all three metals, a clear trend of decreasing modulus with increasing
temperature is observed. Values were calculated with Equation \ref{eq:6}.

For brass, the gradual decrease from 24.3 GPa at 60°C to 23.8 GPa at 180°C shows a slope with a relatively small decline, indicating that the
material maintained much of its torsional rigidity throughout the tested temperature range.

For stainless steel, the highest modulus of rigidity values were observed overall, ranging from 32.0 GPa at 60°C to 31.0 GPa at 180°C. The gradual
decline of values is similar in behaviour to brass.

For aluminium, the modulus was lowest overall and showed the most pronounced decrease, ranging from 12.1 GPa at 60°C to 11.6 GPa at 180°C. The
gradual decreasing behaviour was similar as previously.

\subsection{Internal Friction}

The values for the internal friction were measured and plotted against temperature with Equation \ref{eq:7}. The obtained values are shown in
Table \ref{tab:intfrict} and graphically in Figure \ref{fig:intf}.

Small variations in brass between 0.00608 and 0.00632 remained relatively steady and indicated consistent damping characteristics.

Stainless steel showcased the lowest internal friction within the observed group, with values ranging from 0.00213 to 0.00258. The data showed little temperature dependence, 
indicating stable internal energy dissipation.

Aluminium internal friction values ranged from 0.00190 to 0.00444. The broader spread in the values correspond to the iregular resonance profiles
observed previously.

\begin{figure*}[htbp]
    \centering
    \includegraphics[width=.9\textwidth]{mod rigid graph.png}
    \caption{Graph of the modulus of rigidity for varying temperature for: brass (top), stainless steel (middle), aluminium (bottom).}
    \label{fig:modr}
\end{figure*}

\begin{figure*}[htbp]
    \centering
    \includegraphics[width=.9\textwidth]{intfrict.png}
    \caption{Graph of the internal friction for varying temperature for: brass (top), stainless steel (middle), aluminium (bottom)}
    \label{fig:intf}
\end{figure*}

\section{Analysis and Discussion}

The amplitude-frequency plots obtained for the brass, stainless steel, and aluminium metal rods demonstrated a clear resonant peak. This resonant
frequency value decreased with increasing temperatures across all metals, which is behaviour predicted by theory as elasticity increases with
increase in temperature, leading to a decrease in rigidity and therefore difference in resonant frequency. The approximate values for the
resonant frequency averaged for the varying temperatures were found to be \textbf{194.53 Hz} for brass, \textbf{92.21 Hz} for stainless steel, and
\textbf{178.59 Hz} for aluminium. The expected ordering places stainless steel between brass and aluminium, however no resonance was observed at
those frequencies. This may suggest systematic error or changes to the composition from environmental factors, or incorrect wiring of the experimental
setup.

\textbf{Brass} produced a narrow and symmetric Lorentzian curve shape, which indicated relatively low damping and consistent internal energy losses across
the experimentally observed temperature range. The full width at half maximum remained almost constant, which therefore implied that the Internal
friction did not vary with temperature. The small decrease of the modulus of rigidity for the relative values obtained showcase a torsional stability
for the metal.

\textbf{Stainless steel} showcased the tallest and most narrow resonance peaks when fitted with the Lorentzian, which indicated low damping and high torsional
oscillation quality. The internal friction values measured were the lowest, which support this interpretation of results, with its modulus of rigidity
decreasing between only 32.0 GPa and 31.0 GPa with temperature variations, implying strong thermal resilience. However, the resonant frequency achieved
for this metal was much lower than expected, suggesting that the effective torsional constant of the specimen rod was reduced. This discrepancy may
be attributed to clamping inconsistencies, magnet misalignment, or imperfections in the rod material and shape.

\textbf{Aluminium} exhibited the broadest resonant frequency curves, with observationally decreasing amplitude with increasing temperature. This may
suggest the material's poor ability to sustain oscillations, increased damping, and higher sensitivity of the modulus or rigidity for varying temperatures.
The reduction in the modulus of rigidity, when accompanied by the values obtained for the internal friction, are consistent with the broader, loss
sharply defined resonance peaks observed in the graphs.

The trend observed across all metals for the gradual decline in the modulus of rigidity with increasing temperature align with theory, in which atomic
vibrations in the material due to temperature produce poorly coordinated packings which reduce the shear modulus. Internal friction showed weaker
dependence on the temperature, with brass remaining relatively constant, stainless steel remained low with minor fluctuation, and stainless steel displayed
irregularity consistent with the behaviour observed with a broader bandwidth and increased internal friction at higher temperature.

Sources of error in experimentation include misalignment of the mirror-laser path, too-high gain given to the voltage that produces amplitude
greater than the sensor is able to detect resulting in cut-off values (dips shown in Fig. \ref{fig:a2} for lower temperatures), non-uniform heating
by the heating coil, calibration errors in the voltage-frequency conversion, environmental temperatures interacting with circuitry, or variability in 
torque in the specimen rods by the initial taps. Furthermore, the experimental values obtained for the modulus of rigidity were significantly lower
than published values (Table \ref{tab:modrig}), indicating losses due to experimental constraints, imperfect rod geometry, residual damping from the
support stand and clamp, or errors in calculations and coding. Higher temperature readings for the metals could have been taken with proper 
equipment, which would offer a better look at the behaviour of the composition, specifically with the internal friction calculations.

Overall, despite deviations in absolute values, the trends observed from the data obtained for resonance behaviour, modulus of rigidity, and internal
friction meaningfully reflect the material properties.

\section{Conclusion}

The experiment successfully demonstrated how temperature influences the rigidity and energy dissipation of brass, stainless steel, and aluminium
through a torsional oscillator setup. Across all samples, the resonant frequency and shear modulus gradually decreased with increasing temperatures,
showcasing the behaviour of atomic packing weakening with thermal influence. Stainless steel exhibited the highest modulus of rigidity and lowest 
internal friction across the three specimen, indicating superior torsional stability. Aluminium showcased the most pronounced damping behaviour and
lowest modulus of rigidity. Brass showcased intermediate behaviour with consistent resonance profiles and minimal variations in internal friction values.

Although the achieved absolute values for the modulus of rigidity were significantly lower than those expected from published literature, the observed
trends were physically consistent. Discrepancies likely arose from experimental limitations such as imperfections on the specimen rod, uneven heating,
mirror-laser minor misalignments, incorrect gain provided, and damping from the clamping. Despite these factors, the experiment clearly graphically
showcased the temperature dependence on torsional oscillations, confirming theoretical expectations and demonstrating how material microstructures
and composition influence internal friction. Further improvements in apparatus calibration and temperature control would refine measurement accuracy and
extend investigation on the factors discussed.

%%%%%%%%%%%%%%%%%%%%

% References - switch to single column
\onecolumn

\bibliographystyle{IEEEtran}
\bibliography{PHYC30170References} \label{sec:ref}

\newpage

% Appendix - single column
\section*{Appendix} \label{sec:A}
\addcontentsline{toc}{section}{Appendix}

\begin{table}[H]
\centering
\caption{Table for the obtained and expected values of the modulus of rigidity for brass, stainless steel, aluminium in GPa.}
\label{tab:modrig}
\begin{tabular}{c|ccccc|c}
    \cline{2-6}
                                                                & \multicolumn{5}{c|}{\cellcolor[HTML]{C0C0C0}Modulus of Rigidity (GPa)}                                                                                                                                                  &                                                             \\ \hline
    \rowcolor[HTML]{EFEFEF} 
    \multicolumn{1}{|c|}{\cellcolor[HTML]{C0C0C0}Metal}           & \multicolumn{1}{c|}{\cellcolor[HTML]{EFEFEF}60°C} & \multicolumn{1}{c|}{\cellcolor[HTML]{EFEFEF}90°C} & \multicolumn{1}{c|}{\cellcolor[HTML]{EFEFEF}120°C} & \multicolumn{1}{c|}{\cellcolor[HTML]{EFEFEF}150°C} & 180°C & \multicolumn{1}{c|}{\cellcolor[HTML]{C0C0C0}Expected (GPa)} \\ \hline
    \multicolumn{1}{|c|}{\cellcolor[HTML]{EFEFEF}Brass}           & \multicolumn{1}{c|}{24.3}                         & \multicolumn{1}{c|}{24.2}                         & \multicolumn{1}{c|}{24.1}                          & \multicolumn{1}{c|}{24.0}                          & 23.8  & \multicolumn{1}{c|}{40.0}                                   \\ \hline
    \multicolumn{1}{|c|}{\cellcolor[HTML]{EFEFEF}Stainless Steel} & \multicolumn{1}{c|}{32.0}                         & \multicolumn{1}{c|}{31.9}                         & \multicolumn{1}{c|}{31.6}                          & \multicolumn{1}{c|}{31.4}                          & 31.0  & \multicolumn{1}{c|}{78.0}                                   \\ \hline
    \multicolumn{1}{|c|}{\cellcolor[HTML]{EFEFEF}Aluminium}       & \multicolumn{1}{c|}{12.1}                         & \multicolumn{1}{c|}{12.0}                         & \multicolumn{1}{c|}{11.9}                          & \multicolumn{1}{c|}{11.8}                          & 11.6  & \multicolumn{1}{c|}{26.6}                                   \\ \hline
\end{tabular}
\end{table}


\begin{table}[H]
\centering
\caption{Table for the obtained values of the internal friction for brass, stainless steel, aluminium.} \vspace{0.5cm}
\label{tab:intfrict}
\begin{tabular}{c|ccccc|}
\cline{2-6}
                                                              & \multicolumn{5}{c|}{\cellcolor[HTML]{C0C0C0}Internal Friction}                                                                                                                                                            \\ \hline
\rowcolor[HTML]{EFEFEF} 
\multicolumn{1}{|c|}{\cellcolor[HTML]{C0C0C0}Metal}           & \multicolumn{1}{c|}{\cellcolor[HTML]{EFEFEF}60°C} & \multicolumn{1}{c|}{\cellcolor[HTML]{EFEFEF}90°C} & \multicolumn{1}{c|}{\cellcolor[HTML]{EFEFEF}120°C} & \multicolumn{1}{c|}{\cellcolor[HTML]{EFEFEF}150°C} & 180°C   \\ \hline
\multicolumn{1}{|c|}{\cellcolor[HTML]{EFEFEF}Brass}           & \multicolumn{1}{c|}{0.00632}                      & \multicolumn{1}{c|}{0.00615}                      & \multicolumn{1}{c|}{0.00608}                       & \multicolumn{1}{c|}{0.00619}                       & 0.00630 \\ \hline
\multicolumn{1}{|c|}{\cellcolor[HTML]{EFEFEF}Stainless Steel} & \multicolumn{1}{c|}{0.00258}                      & \multicolumn{1}{c|}{0.00213}                      & \multicolumn{1}{c|}{0.00237}                       & \multicolumn{1}{c|}{0.00237}                       & 0.00243 \\ \hline
\multicolumn{1}{|c|}{\cellcolor[HTML]{EFEFEF}Aluminium}       & \multicolumn{1}{c|}{0.00328}                      & \multicolumn{1}{c|}{0.00321}                      & \multicolumn{1}{c|}{0.00190}                       & \multicolumn{1}{c|}{0.00278}                       & 0.00444 \\ \hline
\end{tabular}
\end{table}

\includepdf[pages=1-42]{/Users/JoanaUCD/Library/CloudStorage/OneDrive-UniversityCollegeDublin/Labs/labs-files/tex Files/modex_graphcode.pdf}


\end{document}