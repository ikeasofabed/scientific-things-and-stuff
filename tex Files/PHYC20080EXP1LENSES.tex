\documentclass[12pt]{article}
\usepackage[a4paper, margin=2cm]{geometry}
\usepackage[english]{babel} % To obtain English text with the blindtext package
\usepackage{blindtext}
\usepackage{graphicx} % Required for inserting images
\usepackage{array} % For extra column formatting
\usepackage{amsmath} %for equation environment
\usepackage{float}
\usepackage{parskip} % For gaps between para
\usepackage{setspace}
\usepackage{pdfpages}
\usepackage{abstract}
\usepackage[export]{adjustbox}
\usepackage{emptypage}
\usepackage{tocloft}
\usepackage[nottoc]{tocbibind}
\usepackage{hyperref, url}
\usepackage{subcaption}
\usepackage{lipsum}
\usepackage{xcolor}
\usepackage{caption}
    \captionsetup{font=footnotesize,labelfont=bf}

\cftsetindents{section}{0em}{2em}
\cftsetindents{subsection}{0em}{2em}

\renewcommand\cfttoctitlefont{\hfill\Large\bfseries}
\renewcommand\cftaftertoctitle{\hfill\mbox{}}

\graphicspath{ {./images/} }

\pagenumbering{arabic}

\definecolor{blurple}{HTML}{5865F2}

\hypersetup{
    colorlinks=true,
    linkcolor=black,
    urlcolor=blurple,
    citecolor=blurple,
}

\urlstyle{same}

\renewcommand{\arraystretch}{1.3}

%%%%%%%%%%%%%%%%%%%%%%%%%%%%%%%%%%%


\title{PHYC20080 Exp.1 Lenses}
\author{Joana Adao}
\date{\today}

\begin{document}

\begin{titlepage}
    \begin{center}

        \begin{figure}[ht]
            \includegraphics[width=\textwidth]{UCDLogo.png}
        \end{figure}
        
        \begin{figure}
            \centerline{\includegraphics[width=\paperwidth]{UCDBanner.png}}
        \end{figure}

        \vspace{4cm}

        {\LARGE \bfseries PHYC20080 Fields, Waves and Light}\\
        \vspace{0.75cm}
        {\Large Experiment No.1 Measurement of the Focal Lengths of Lenses and a Determination of Brewster's Angle}
        
        \vspace{1cm}
    
    {\Large \textbf{4 February 2025}}

    \vspace{2cm}
    
    {\large \textbf{by Joana C.C. Adao (Student No. 23311051)}}\\
    \vspace{0.25cm}
    {\large Tuesday 16.00-18.00 Slot}\\
    {\large Nicki (Coordinator)}

    \end{center}
    
   \clearpage

\end{titlepage}

\setcounter{page}{1}
\tableofcontents

\newpage

\begin{abstract}
\addcontentsline{toc}{section}{Abstract}

\vspace{1cm}

The aim of this experiment 

\end{abstract}

%%%%%%%%%%%%%%%%%%%%%%%%%%%%%%%%%%%

\section{Theory} \label{sec:1}


\subsection{Lenses} \label{sec:1.1}

Lenses are a piece of transparent materials, typically glass, that are used to bend and focus rays of light onto one spot to form images
\cite{britlens,vedantulens}.
Lenses can either be shaped to be concave (depressed, caved) or convex (bulging, rounded), but have to have at least one curved surface
\cite{britlens,vedantulens}
The properties that make up a lens are: \textit{focal length} (§\ref{sec:1.1.1}), \textit{focal point}, \textit{optical axis}, \textit{focus}, \textit{principal axis}, \textit{centre of curvature}
\cite{geekconcave,geekconvex}

\begin{figure}[H]
    \centering
    \includegraphics[width=15cm]{lenses.png}
    \caption{\centering Diagram of converging (convex) and diverging (concave) lenses \protect\cite{britlens}.
    \newline
    \centering \scriptsize{\textit{(a): biconvex, (b): plano-convex, (c): positive meniscus; (d): biconcave, (e): plano-concave, (f): negative meniscus}}}
    \label{fig:lens}
\end{figure}

Convex lenses make the parallel travelling rays of light passing through it \textbf{converge} to a certain point known as the \textit{focal point} (\ref{sec:1.1.1})
\cite{studyconvexlens}.
The degree at which light is bent depends on the absolute curvature of the lens
\cite{shanghaiconvex}.
The convex lens has two extra properties that do not apply to a concave lens: \textit{radius of curvature}, and \textit{aperture}
\cite{geekconvex}.
The behaviour of ligh as it passes through a convex lens can be simulated via the interactive diagram link in \textit{References}, §\ref{sec:ref}
\cite{convexinteract}.

Concave lenses make the parallel travelling rays of ligth passing through it \textbf{refract} away from the focal point of the lens
\cite{studyconvexlens,studyconcavelens}.
The behaviour of ligh as it passes through a concave lens can be simulated via the interactive diagram link in \textit{References}, §\ref{sec:ref}
\cite{concaveinteract}.

\subsubsection{Focal Length} \label{sec:1.1.1}

The focal length is the distance, typically in millimetres, of the \textit{optical axis} (the centre of the lens) to the focal plane shown in figure \ref{fig:lens}
\cite{canonfocal,studyfocal}.
The focal point, indicated by the point at which the focal plane and principal axis intersect, is the point at which the rays of light coincide
\cite{isaaclens}.
For a concave lens, where the light diverges, there only exists an \textit{apparent} focal point behind the lens, where a virtual image would be formed (§\ref{sec:1.1.2})
\cite{isaaclens}.

The formula below, equation \ref{eq:1}, is used to calculate the focal lenght of a thin converging lens \cite{UCDlens,isaaclens}, where \textbf{u} is the distance
between the \textit{object} and the centre of the lens, and \textbf{v} is the distance between the \textit{image} and the centre of the lens:

\begin{gather} \label{eq:1}
    \frac{1}{f} = \frac{1}{u} + \frac{1}{v}
\end{gather}

Since a concave lens does not form a real image (§\ref{sec:1.1.2}) one must add a convex lens to overcome this issue. Therefore equation \ref{eq:1} becomes invalid (individually) for calculating the focal length.
Instead, equation \ref{eq:2} below can be used to find the focal length of the combined lenses, where $f_{convex}$ (focal length of the convex lens) and $f_{concave}$ (focal length of the concave lens) can be calculated individually
using equation \ref{eq:1} \cite{UCDlens}:  

\begin{gather} \label{eq:2}
    \frac{1}{f} = \frac{1}{f_{convex}} + \frac{1}{f_{concave}}
\end{gather}

Concave lenses have \textit{negative} focal length, as opposed to convex lenses, since the focal point is on the same side as the incident (incoming) light \cite{geekconcave}.

\subsubsection{Image Formation} \label{sec:1.1.2}

Two types of images can be formed when looking at a mirror or through a lens:
\begin{itemize}
    \item \textbf{Real image:} an image formed when light converges at a point; inverted in nature \cite{geekrelvir}.
    \item \textbf{Virtual image:} an image formed when light diverges away from a point; appears to be produced but is not actually present \cite{geekrelvir}.
\end{itemize}

Standard image properties of convex lenses is that what is produced is a real image that is inverted and enlarged
\cite{geekconvex}, but virtual images can also be formed.
The image produced depends on the object position at the time of measurement.
The types of images convex lenses can form are illustrated in figure \ref{fig:conveximage} and described in table \ref{tab:1} (\textit{sourced directly from Geeksforgeeks \cite{geekconvex}}).

\begin{minipage}{.5\textwidth}
    \captionsetup{hypcap=false}
    \centering
    \includegraphics[width=.9\linewidth]{convex.jpg}
    \captionof{figure}{\centering Ray diagram of the cases of convex lens image formation \protect\cite{topprlensimage}.}
    \label{fig:conveximage}
\end{minipage}
\hfill
\begin{minipage}{.45\textwidth}
    \captionsetup{hypcap=false}
    \centering
    
    \begin{table}[H]
        \centering
        \resizebox{.9\linewidth}{!}{

            \begin{tabular}{|c|c|c|c|}
            \hline
            \textbf{\begin{tabular}[c]{@{}c@{}}Object\\ Position\end{tabular}}     & \textbf{\begin{tabular}[c]{@{}c@{}}Image\\ Position\end{tabular}} & \textbf{\begin{tabular}[c]{@{}c@{}}Image\\ Size\end{tabular}} & \textbf{\begin{tabular}[c]{@{}c@{}}Image\\ Nature\end{tabular}}                 \\ \hline
            \textbf{Beyond 2F}                                                     & \begin{tabular}[c]{@{}c@{}}Between \\ F and 2F\end{tabular}       & Smaller                                                       & \begin{tabular}[c]{@{}c@{}}Real,\\ Inverted\end{tabular}                        \\ \hline
            \textbf{At 2F}                                                         & At 2F                                                             & Same size                                                     & \begin{tabular}[c]{@{}c@{}}Real, \\ Inverted\end{tabular}                       \\ \hline
            \textbf{\begin{tabular}[c]{@{}c@{}}Between \\ F and 2F\end{tabular}}   & Beyond 2F                                                         & Larger                                                        & \begin{tabular}[c]{@{}c@{}}Real, \\ Inverted\end{tabular}                       \\ \hline
            \textbf{At F}                                                          & Infinity                                                          & Infinite                                                      & \begin{tabular}[c]{@{}c@{}}Real, \\ Inverted\\ (Highly Diminished)\end{tabular} \\ \hline
            \textbf{\begin{tabular}[c]{@{}c@{}}Between \\ F and lens\end{tabular}} & Beyond 2F                                                         & Larger                                                        & \begin{tabular}[c]{@{}c@{}}Virtual, \\ Upright\end{tabular}                     \\ \hline
            \textbf{At lens}                                                       & At lens                                                           & Larger                                                        & \begin{tabular}[c]{@{}c@{}}Virtual, \\ Upright\end{tabular}                     \\ \hline
            \textbf{\begin{tabular}[c]{@{}c@{}}Object inside \\ lens\end{tabular}} & \begin{tabular}[c]{@{}c@{}}Between \\ lens and F\end{tabular}     & Larger                                                        & \begin{tabular}[c]{@{}c@{}}Virtual, \\ Upright\end{tabular}                     \\ \hline
            \end{tabular}

        }
        \caption{\centering Table of the different cases of convex lens image formation \protect\cite{geekconvex,topprlensimage}.}
        \label{tab:1}
    \end{table}
\end{minipage}

Standard image properties of concave lenses is that what is produced is a virtual image that is upright and diminished
\cite{geekconcave}.
Concave lenses cannot produce real images. The only thing that changes between the virtual images that are produced is the magnitude at which they're diminished
\cite{topprlensimage,geekconcave}.

\begin{minipage}{.5\textwidth}
    \captionsetup{hypcap=false}
    \centering
    \includegraphics[width=.9\linewidth]{concave.png}
    \captionof{figure}{\centering Ray diagram of concave lens image formation \protect\cite{topprlensimage}.}
    \label{fig:concaveimage}
\end{minipage}
\hfill
\begin{minipage}{.45\textwidth}
    \captionsetup{hypcap=false}
    \centering
    
    \begin{table}[H]
        \centering
        \resizebox{.9\linewidth}{!}{

            \begin{tabular}{|c|c|c|c|}
            \hline
            \textbf{\begin{tabular}[c]{@{}c@{}}Object\\ Position\end{tabular}}     & \textbf{\begin{tabular}[c]{@{}c@{}}Image\\ Position\end{tabular}} & \textbf{\begin{tabular}[c]{@{}c@{}}Image\\ Size\end{tabular}}            & \textbf{\begin{tabular}[c]{@{}c@{}}Image\\ Nature\end{tabular}} \\ \hline
            \textbf{At infinity}                                                   & At $F_2$                                                          & \begin{tabular}[c]{@{}c@{}}Highly Diminished,\\ Point-Sized\end{tabular} & \begin{tabular}[c]{@{}c@{}}Virtual, \\ Upright\end{tabular}     \\ \hline
            \textbf{\begin{tabular}[c]{@{}c@{}}Object inside \\ lens\end{tabular}} & \begin{tabular}[c]{@{}c@{}}Between \\ lens and $F_2$\end{tabular} & Diminished                                                               & \begin{tabular}[c]{@{}c@{}}Virtual, \\ Upright\end{tabular}     \\ \hline
            \end{tabular}

        }
        \caption{\centering Table of the different cases of concave lens image formation \protect\cite{geekconcave,topprlensimage}}
        \label{tab:2}
    \end{table}
\end{minipage}

\subsection{Polarisation}

Since light has both particle and (transverse) wave properties, it can be restricted to propagating in one plane, shown in figure \ref{fig:polar}
\cite{isaacpolar,britpolar}.

\begin{figure}[H]
    \centering
    \includegraphics[width=10cm]{polarisation.png}
    \caption{\centering Visual representation of polarisation of light \protect\cite{britpolar}.}
    \label{fig:polar}
\end{figure}

\subsubsection{Snell's Law}



\subsubsection{Brewster's Angle}



\subsection{Reflection and Refraction}



\subsubsection{An Aside on Reflection and Refraction Coefficients}



\section{Methodology} 

\subsection{Determining the Focal Length of a Convex Lense Method}



\subsection{Determining the Focal Length of a Concave Lense Method}



\subsection{Determining Brewster's Angle Meethod}



\section{Results and Calculations}

\subsection{Determining the Focal Length of a Convex Lense}



\subsection{Determining the Focal Length of a Concave Lense}



\subsection{Determining Brewster's Angle}



\section{Conclusion} \label{sec:4}


\section{Applications of Lenses}

\subsection{Eyeglasses}

\subsection{Microscopes}

\subsection{Telescopes}

\newpage

%%%%%%%%%%%%%%%%%%%%%%%%%%%%%%%%%%%

\bibliographystyle{IEEEtran}
\bibliography{References} \label{sec:ref}

\newpage

\section*{Appendix} \label{sec:A}
\addcontentsline{toc}{section}{Appendix}

\listoffigures

\listoftables

\subsection*{Raw Data}
\addcontentsline{toc}{subsection}{Raw Data}



\subsection*{Code}
\addcontentsline{toc}{subsection}{Code}



\end{document}